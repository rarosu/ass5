\documentclass{article}
\author{Lars Woxberg \and Daniel Carlsson \and Philip Heilmair \and Thomas Sievert}
\date{\today}
\title{Assignment 5}

\usepackage[swedish]{babel}
\usepackage{graphicx}
\usepackage{hyperref}
\usepackage{url}
\usepackage[utf8]{inputenc}

\renewcommand{\refname}{Referenser}
\renewcommand{\abstractname}{Abstrakt}

\begin{document}
\maketitle

%\begin{abstract}
%\end{abstract}

\section{Introduktion}
Målet med experimentet är att studera ögonrörelser i spelet SCP: Containment Breach \cite{SCP}, speciellt hur mycket fokus som läggs på spelets \textit{heads-up display} (HUD) gentemot omgivningen. 

SCP: Containment Breach är ett skräckspel sett från ett förstapersonsperspektiv med slumpgenererade nivåer och händelser. Målet med spelet är att fly ut ur ett nedlåst laboratorium där det finns livsfarliga monster som brutit sig ur sin säkra förvaring. Det monster spelaren oftast stöter på kallas SCP-173 (se Figur~\ref{hud}) och kan röra sig mycket fort när hen inte tittar på det. När det syns på skärmen står dock monstret still. Övriga monster har ingen bäring på experimentet och bortses därför från helt och hållet.

Ett stort fokus i spelet är att spelaren måste blinka med jämna mellanrum vilket resulterar i att skärmen blir svart för ett ögonblick. Denna blinkning gör det fritt fram för SCP-173 att röra sig, och om spelaren då står för nära är spelet slut. Hur mycket tid spelaren har kvar tills hen måste blinka syns i HUDen.

\begin{figure}[h!]
    \begin{center}
        \includegraphics*[width=0.63\columnwidth]{experiment/SCP-173-06-aoi.png}
        \caption{Skärmdump med HUDen markerad inom det röda området.}
        \label{hud}
    \end{center}
\end{figure}

Den övre stapeln visar hur mycket tid spelaren har tills hen automatiskt blinkar. Spelaren kan blinka när som helst och återställa tiden tills hen behöver blinka igen.  Den undre stapeln beskriver hur länge hen kan springa och är ointressant för detta experiment.

Blinkmekaniken introducerar ett visst mått av "taktikblinkande" till spelet; det är gynnsamt att blinka ofta när monstret inte är i sikte, för att maximera tiden till nästa blinkning då SCP-173 väl dyker upp. Därför förmedlar blinkmätaren information som blir betydligt viktigare när monstret är i närheten.


\section{Hypotes}
Tanken är att spelaren kommer att hålla mer fokus på HUDen när monstret är nära. Det är intressant att veta var spelaren har fokus när hen möter monstret, så att spelutvecklaren kan få en idé om hur mycket arbete som bör läggas på detaljerna i omgivningen. Det är också intressant att veta hur mycket HUDen används i allmänhet.

\section{Metod}
Första steget är att bestämma \textit{areas-of-interest} på skärmen för var HUDen är. Detta är det område som är intressant att se hur mycket blicken ligger vid. Intresseområdet ligger inom det lilla röda området i Figur~\ref{hud}.

För att samla data spelas en spelomgång in vid en eye-tracker som kan följa ögonen på skärmen. Först får spelaren spela en omgång utan inspelning för att bekanta sig med spelet. Detta är för att hen ska vara bekant med var HUDen finns och vad den innebär. 

Sedan spelas ögonrörelser in i en ny spelomgång. Videon kan sedan undersökas över intressanta sektioner för att se om blicken ligger på HUDen eller på omgivningen.

\section{Resultat}

Fem studenter vid Blekinge Tekniska Högskola deltog i experimentet. Totalt erhölls värdefull data motsvarande omkring tio minuter från försöken.

Datan som samlades in bestod av detaljerad information av var på skärmen spelaren tittade. Relevant data för experimentet hämtades ut i form av värmekartor som visar hur ofta en spelare tittar på ett specifikt område och information om hur mycket tid hen tittade på HUD-elementen kontra resten av skärmen. HUD-elementen utgjorde således de speciella \textit{areas-of-interest} som skulle spåras specifikt.

\begin{figure}[h!]
    \begin{center}
        \includegraphics*[width=0.63\columnwidth]{experiment/NoContact_Heatmap.png}
        \caption{Ingen kontakt med monstret.}
        \label{no_contact}
    \end{center}
\end{figure}

\begin{figure}[h!]
    \begin{center}
        \includegraphics*[width=0.63\columnwidth]{experiment/Contact_Heatmap.png}
        \caption{Kontakt med monstret.}
        \label{contact}
    \end{center}
\end{figure}

\newpage

Värmekartorna gjordes genom att beräkna ett genomsnitt över var spelaren tittade all tid då hen såg, respektive inte såg, monstret. Dessa beskriver tydligt att spelarens blick inte är lika fokuserad när monstret var närvarande. Detta fenomen stärktes ytterliggare av de spårade \textit{areas-of-interest} (se figur~\ref{bars}), som visar att spelaren tittade ungefär fem gånger mindre på HUD-området vid monstrets närvaro.

\begin{figure}[h!]
    \begin{center}
        \includegraphics*[width=0.50\columnwidth]{experiment/bars-grayscale.png}
        \caption{Fokus på HUD.}
        \label{bars}
    \end{center}
\end{figure}

Staplarna i Figur~\ref{bars} beskriver hur länge spelaren tittade inom HUDens område (se det röda området i Figur~\ref{hud}) jämfört med den totala tiden vid närvaro respektive frånvaro av monstret.


\section{Slutsats}

Resultatet motsäger hypotesen eftersom spelaren tittade mindre på HUDen vid monsterkontakt snarare än mer (se figur~\ref{bars}). Dock är varken hypotesen eller datan tillräckliga för att enskilt användas som underlag för beslut kring den visuella designen. Det finns många faktorer som inte har tagits hänsyn till i hypotesen som är viktiga för att kunna göra ett beslut om HUDen, exempelvis hur mycket utvecklarna vill att spelaren ska se HUDen vid monsterkontakt, hur erfaren en spelare är eller hur mycket som är tillräcklig fokustid för att en spelare ska kunna få ut någonting av HUDen.

Vid jämförelse av de två värmekartorna (se figurer~\ref{no_contact} och \ref{contact}) kan ett fenomen noteras som skulle kunna vara både intressant och relevant att undersöka vidare; nämligen att spelarens blick verkar flacka betydligt mer vid monsterkontakt. 

Figur~\ref{bars} är kraftfullast med avseende på hypotesen och besvarar den i sig själv. Diagrammet visar bara förhållandet mellan hur mycket tid som tittades på HUDen jämfört med hela scenen. Det säger ingenting om hur blickerna var utspridda, vilket värmekartorna gör istället. Diagrammet är i gråskala av flera anledningar, dels finns bara två staplar och kontrasten mellan dem särskiljer dem tillräckligt, dels skulle färg oavsiktligt medföra värderingar till staplarna.

Värmekartorna ger mer kontext till experimentet och ger en översikt på hur blicken hos en spelare rör sig. Visualiseringen förlitar sig mer på färgens densitet än på dess kulörthet. Densiteten kan fortfarande uppfattas även med färgblindhet, som kan ses i figurer \ref{heatmap-red-blind} och \ref{heatmap-green-blind}.

\begin{figure}[h!]
    \begin{center}
        \includegraphics*[width=0.50\columnwidth]{experiment/Contact_Heatmap_protanopia.jpg}
        \caption{Rödblindhetssimulering \cite{ColorBlindness}}
        \label{heatmap-red-blind}
    \end{center}
\end{figure}

\begin{figure}[h!]
    \begin{center}
        \includegraphics*[width=0.50\columnwidth]{experiment/Contact_Heatmap_deuteranopia.jpg}
        \caption{Grönblindhetsimulering \cite{ColorBlindness}}
        \label{heatmap-green-blind}
    \end{center}
\end{figure}

Slutligen bör det påpekas att den insamlade mängden data är för liten för att ens kunna fastställa en trend. Då endast fem personer deltog i experimentet, och värdefull data inte kunde utvinnas från allihop innebär det att resultatet lika gärna skulle kunna bero på slumpen.

\bibliographystyle{plain}
\bibliography{assignment-5}

\end{document}

