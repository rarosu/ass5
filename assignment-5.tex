\documentclass{article}
\author{Lars Woxberg \and Daniel Carlsson \and Philip Heilmair \and Thomas Sievert}
\date{\today}
\title{Assignment 5}

\usepackage[swedish]{babel}
\usepackage{graphicx}
\usepackage{hyperref}
\usepackage{url}
\usepackage[utf8]{inputenc}

\renewcommand{\refname}{Referenser}
\renewcommand{\abstractname}{Abstrakt}

\begin{document}
\maketitle

%\begin{abstract}
%\end{abstract}

\section{Introduktion}
Målet med experimentet är att studera ögonrörelser i spelet SCP: Containment Breach \cite{SCP}, speciellt hur mycket fokus som läggs på HUD jämtemot omgivningen. SCP: Containment Breach är ett skräckspel med slumpgenererade nivåer och händelser. Ett stort fokus i spelet är att man måste blinka med jämna mellanrum och att fiender kan röra sig när man blinkar. Hur mycket tid man har kvar tills man måste blinka syns i HUDen och en spelare måste hålla koll på det.

\section{Hypotes}
Tanken är att spelaren kommer att hålla mer fokus på HUDen när en fiende är nära. Det är intressant att veta var spelaren har fokus när man möter en fiende, så att man vet hur mycket arbete man ska lägga på detaljerna i omgivningen. Det är också intressant att veta hur mycket HUDen används i allmänhet.

\section{Metod}
Första steget är att bestämma areas-of-interest på skärmen för var HUDen är. Detta är området som är intressant att se hur mycket blicken ligger vid.

För att samla data så spelas en spelomgång in vid en eye-tracker som kan följa ögonen på skärmen. Först så får spelaren spela en omgång utan inspelning för att bekanta sig med spelet. Detta är för att spelaren ska vara bekant med var HUDen finns och vad den innebär. 

Sedan så spelas ögonrörelser in i en ny spelomgång. Videon kan sedan undersökas över intressanta sektioner för att se om blicken ligger på HUDen eller på omgivningen.

\section{Resultat}

\section{Slutsats}

\bibliographystyle{plain}
\bibliography{assignment-5.bib}

\end{document}

